\documentclass[a4paper]{report} % -*- coding: latin-1 -*-

\usepackage{xspace, palatino, hyperref, graphicx}
\usepackage[latin1]{inputenc}

\newcommand{\pyblio}{\textsl{pybliographer}\xspace}
\newcommand{\py}[1]{\texttt{#1}}
\newcommand{\file}[1]{\texttt{#1}}

\hyphenation{bib-li-o-gra-phic}

\title{Writing bibliographic tools with \pyblio}
\author{Fr�d�ric Gobry}

\begin{document}
\maketitle

\tableofcontents

\chapter{Introduction}

\pyblio   is   a   developer-oriented   framework   for   manipulating
bibliographic  data.  It  is written  in  \textsl{python}\footnote{see
  \url{http://python.org/}}, and  uses extensively the  dynamic nature
of this language.

\pyblio  does   not  try  to   define  another  standard   format  for
bibliographic  data, nor  does it  solely  rely on  a single  existing
standards.    Standards  are   important   in  order   to  allow   for
interoperability and durability.  Unfortunately, real-world data often
contain  a  great  number   of  mistakes,  or  reflect  certain  local
conventions. \pyblio is on  the \textit{pragmatic} side of considering
these issues as part of its business: most of the parsing tasks can be
easily overriden and  specialized in order to \textit{fit  the code to
  the data}, and not the other way around.


\section{Basic concepts}

\pyblio  deals  with  sets  of  \py{Record}s, stored  in  a  so-called
\py{Database}.  This database can  be actually  implemented on  top of
different systems. Two are available  today, one based on a single XML
file,  using  a  custom  XML  dialect, the  other  based  on  Berkeley
DB\footnote{see  \url{http://www.sleepycat.com/}},  a  very  efficient
database system.

Each record represents an elementary  object you want to describe, and
has  a  number  of  \textsl{attributes}.   For instance,  if  you  are
describing a  book, one attribute will be  its \textsl{title}, another
its \textsl{ISBN},  etc. Each of  these attributes can contain  one or
more  values,  all  of   the  same  \textsl{type}.   To  continue  the
description  of  our  book,   we  probably  have  the  \textsl{author}
attribute,  which contains  as many  \py{Person} values  as  there are
authors for the  book. All the values of a given  attribute are of the
same type.

In some cases, simply having  this flat key/value model to describe an
object  is not  enough.   \pyblio  allows, for  every  value of  every
attribute, to  provide a set of  \textsl{qualifiers}. These qualifiers
are also attributes which can hold one or more values.  If my book, or
information  about the  book, is  available  via the  internet, I  can
provide  a  \textsl{link}  attribute,  but  for  each  of  the  actual
\py{URL}s  provided,  I  might  wish  to  add  a  \textsl{description}
qualifier, which will indicate, say, if the URL points to the editor's
website, or to a review, etc.

This nesting of objects is best described in
figure~\ref{fig:hierarchy}.

\begin{figure}[htbp]
  \centering
  \includegraphics[width=0.9\textwidth]{fig/hierarchy.pdf}
  \caption{Objects manipulated in \pyblio}
  \label{fig:hierarchy}
\end{figure}

\pyblio comes with a set of defined attribute types, like \py{Person},
\py{Text}, \py{Date}, \py{ID} (see the \py{Pyblio.Attribute} module
for a complete list), and can be extended to support your own types.


\subsection{The database schema}
\label{sec:schema}

Even though  attributes are typed,  the data model described  above is
quite flexible.  In  order for \pyblio to help  you checking that your
records are properly  typed, it needs to know  the database schema you
are  using.  This  schema,  usually stored  in  an XML  file with  the
extension \file{.sip},  simply lists  the known attributes  with their
type and  the qualifiers it  allows for its values.   Some \file{.sip}
files  are  distributed   with  \pyblio,  and  can  be   seen  in  the
\py{Pyblio.RIP} directory.

In   addition   to  validation   information,   the  schema   contains
human-readable  description  of  the  different  fields,  possibly  in
several languages, so  that it can be automatically  extracted by user
interfaces to provide up-to-date information.

\subsection{Taxonomies}
\label{sec:taxonomies}

TODO


\subsection{Result sets}
\label{sec:resultsets}

TODO



\section{Manipulating data}
\label{sec:manipulating}

TODO


\subsection{Sorting}
\label{sec:sorting}

TODO


\subsection{Searching}
\label{sec:searching}

TODO


\section{Importing and exporting}
\label{sec:importexport}

TODO


\section{Citation formatting}
\label{sec:citation}

TODO



\chapter{Extending \pyblio}

TODO


\section{Specializing a parser}
\label{sec:specializing}

TODO


\end{document}
