\documentclass[a4paper]{report} % -*- coding: latin-1 -*-

\usepackage{xspace, palatino, graphicx, verbatim, varioref}
\usepackage[latin1]{inputenc}
\usepackage[pdftex]{hyperref}

\newcommand{\pyblio}{\textsl{pybliographer}\xspace}
\newcommand{\py}[1]{\texttt{#1}}
\newcommand{\file}[1]{\texttt{#1}}

\hyphenation{bib-li-o-gra-phic}

\title{Writing bibliographic tools with \pyblio}
\author{Fr�d�ric Gobry}

\begin{document}
\maketitle

\tableofcontents

\chapter{Introduction}

\pyblio   is   a   developer-oriented   framework   for   manipulating
bibliographic  data.  It  is  written in  \textsl{python}\footnote{see
  \url{http://python.org/}}, and  uses extensively the  dynamic nature
of this language.

\pyblio  does   not  try  to   define  another  standard   format  for
bibliographic  data, nor  does it  solely  rely on  a single  existing
standards.    Standards  are   important   in  order   to  allow   for
interoperability and durability.  Unfortunately, real-world data often
contain   mistakes   (sometimes   systematic   mistakes   due   to   a
misunderstanding of the meaning of  a field), or reflect certain local
conventions  which are  not  part of  a  standard. \pyblio  is on  the
\textit{pragmatic}  side of considering  these issues  as part  of its
business: most of the parsing  and processing tasks it performs can be
easily overridden and specialized in  order to \textit{fit the code to
  the data}, and not the other way around.


\section{Basic concepts}

\pyblio  deals  with  sets  of  \py{Record}s, stored  in  a  so-called
\py{Database}.  This  database can be  actually implemented on  top of
different systems.  Three are available today:
\begin{itemize}
\item in-memory: useful when the  data is converted from one format to
  another and doesn't need to be stored in \pyblio.
\item file: as a single XML file, using a custom XML dialect, suitable
  for small to medium databases (thousands of records).
\item Berkeley  DB\footnote{see \url{http://www.sleepycat.com/}}: this
  is   a  very   efficient  database   system,  suitable   for  larger
  databases.  In this  case the  limitations are  due to  some \pyblio
  design decisions, and  should be reached after a  million records or
  so.
\end{itemize}

Each record represents an elementary  object you want to describe, and
has  a  number  of  \textsl{attributes}.   For instance,  if  you  are
describing a  book, one attribute will be  its \textsl{title}, another
its \textsl{ISBN},  etc. Each of  these attributes can contain  one or
more  values,  all  of   the  same  \textsl{type}.   To  continue  the
description  of  our  book,   we  probably  have  the  \textsl{author}
attribute,  which contains  as many  \py{Person} values  as  there are
authors for the  book. All the values of a given  attribute are of the
same type.

In some cases, simply having  this flat key/value model to describe an
object is not  enough.  \pyblio allows every value  of every attribute
to  provide a set  of \textsl{qualifiers}.  These qualifiers  are also
attributes  which  can  hold one  or  more  values.   If my  book,  or
information  about the  book, is  available  via the  internet, I  can
provide  a  \textsl{link}  attribute,  but  for  each  of  the  actual
\py{URL}s  provided,  I  might  wish  to  add  a  \textsl{description}
qualifier, which will indicate, say, if the URL points to the editor's
website, or to a review, etc.

This nesting of objects is best described in
figure~\ref{fig:hierarchy}.

\begin{figure}[htbp]
  \centering
  \includegraphics[width=0.9\textwidth]{fig/hierarchy.pdf}
  \caption{Objects manipulated in \pyblio}
  \label{fig:hierarchy}
\end{figure}

\pyblio comes with a set of defined attribute types, like \py{Person},
\py{Text},  \py{Date}, \py{ID}  (see the  \py{Pyblio.Attribute} module
for a complete list), and can be extended to support your own types.


\subsection{The database schema}
\label{sec:schema}

Even though  attributes are typed,  the data model described  above is
quite flexible.  In  order for \pyblio to help  you checking that your
records are properly  typed, it needs to know  the database schema you
are  using.  This  schema,  usually stored  in  an XML  file with  the
extension \file{.sip},  simply lists  the known attributes  with their
type and  the qualifiers it  allows for its values.   Some \file{.sip}
files  are  distributed   with  \pyblio,  and  can  be   seen  in  the
\py{Pyblio.RIP} directory.

In   addition   to  validation   information,   the  schema   contains
human-readable  description  of  the  different  fields,  possibly  in
several languages, so  that it can be automatically  extracted by user
interfaces to provide up-to-date information.

\subsection{Taxonomies}
\label{sec:taxonomies}

Taxonomies can be used as \textit{enumerated values}, say for listing
the possible types of a document, or the language in which a text is
written. They have however the extra capability of being hierarchical:
you can define subcategories of a main category.  For instance,
imagine a \py{doctype} taxonomy with the following values:

\begin{center}
  \includegraphics[width=0.7\textwidth]{fig/tree.pdf}
\end{center}

You can tag an article as \py{Peer-reviewed}, but you are not required
to use  the \textit{leaf} values in  this tree. In the  case you don't
know if a publication is reviewed or not, you can use the \py{Article}
tag. Similarly,  if you search  for all the  \py{Published} documents,
you will retrieve all those that have the \py{Published} tag, but also
those that are articles (either peer-reviewed or not), books,...

\pyblio     uses    the     \py{Pyblio.Attribute.Txo}     object    to
\textit{represent} a logical  value in a given taxonomy.  A record can
be    tagged    with    this    \py{Txo}   object    by    adding    a
\py{Pyblio.Schema.TxoItem} value in the corresponding attribute.

Taxonomies are declared  in a database schema, and  thus cannot change
unless you change the schema itself.

\subsection{Result sets}
\label{sec:resultsets}

Result sets are used to  manipulate an explicit list of records, among
all the records kept in a  database. They are returned from queries on
the  database, and can  be manipulated  by the  user. Result  sets are
somewhat like mathematical sets, as you cannot put duplicate values in
them, and  they have  no default ordering  of their elements.  You can
create result sets  via the \py{rs} attribute of  your database, which
is an instance of the \py{Pyblio.Store.ResultSetStore}.

A special result set is available as \py{Pyblio.entries}, and contains
at every time \textbf{all} the records of the database.


\subsection{Views}
\label{sec:views}

We have  seen that result  sets are \textbf{not} ordered.  However, in
many cases, one  needs to provide the records in  a specific order. To
do so,  you can create  a \textsl{view} on  top of a result  set. This
view is  created by  calling the \py{view}  method of the  result set,
with an \py{order}  parameter being the description of  the sort order
you  wish to  have.  The module  \py{Pyblio.Sort} provides  elementary
constructs to build such a description.

Once the view is created, modifying the corresponding result set leads
to updating the view accordingly.

\section{Manipulating data}
\label{sec:manipulating}

This section describes some simple  operations you can perform on some
subset of a \pyblio database.

\subsection{Loading and saving}
\label{sec:loading}

The first thing you need to do is of course \textit{actually having} a
database available. The following code does the job:
\verbatiminput{code/create.py}

This example  relies on  the fact  that you already  have a  schema at
hand. There are schemas  available in the \file{Pyblio.RIP} directory.
It the  starts by reading the schema.  The next step is  to select the
actual  physical store  which will  hold your  database. We  choose to
store  it  in  a  simple   XML  file,  whose  canonical  extension  is
\file{.bip}. The last operation actually creates the database with the
specified schema.

Independently  of  the  selected  store,  it  is  always  possible  to
\textit{export} a  database in the \file{.sip} format,  by calling the
\py{db.xmlwrite(...)} method of the database.  Such a file can then be
reused   later  on  by   using  \py{store.dbimport(...)}   instead  of
\py{store..dbcreate(...)}.

When  you  have  finished   modifying  your  database,  you  can  call
\py{db.save()} method to ensure that it is properly saved.

\textbf{Caution:}  the bsddb store  for instance  is updated  at every
actual  modification, not  only when  you call  the  \py{save} method.
Don't rely on it to provide some kind of \textsl{rollback} feature.

\subsection{Using the registry}
\label{sec:registry}

\pyblio has a  mechanism to register known schemas,  and specify which
import and  export filters  can properly work  with each  schema. This
mechanism can be used to create  our database by asking for a specific
schema, as shown below: \verbatiminput{code/registry.py}

The  registry  must be  first  initialized. Then  you  can  ask for  a
specific schema, in that case a schema that supports BibTeX databases.

\subsection{Updating records}
\label{sec:updating}

The next example will loop over all the records in a database, and add
a      new      author      to      the     list      of      authors.
\verbatiminput{code/addauthor.py}

We use  the \py{itervalues()}  iterator to loop  over all  the records
stored in  the database.  Then, we  simply insert a  new value  in the
\py{author} attribute.  The \py{record.add(...)} method  takes care of
creating the attribute if it does not exist yet.

One thing  not to forget is to  store the record back  in the database
once  the modification  is  performed. Without  this  step, you  might
experience weird  behavior where  some modifications are  not properly
kept.

We finish by saving the database.

\subsection{Sorting}
\label{sec:sorting}

To     sort     records,     you    create     \textit{views}     (see
section~\vref{sec:views}). You can of  course create multiple views on
top  of a  single result  set. In  order to  sort the  whole database,
simply create  the view on  \py{database.entries} instead of  a result
set. If you want to sort  your database by decreasing year and then by
author, you can use a view like that:

\verbatiminput{code/sort.py}

So, sorting  constraints can be  arbitrarily chained with  the \py{\&}
operator,and  each constraint  can be  either  \textit{ascending} (the
default),  or   \textit{descending}.   This  defines   a  very  simple
\textsl{Domain Specific  Language}, or  DSL for short.  Such languages
also appear in other part of \pyblio (searching, citation formatting),
as they are  a convenient way to describe  complex abstraction without
having to reinvent a complete environment.

\subsection{Searching}
\label{sec:searching}

To search,  you call the \py{database.query(...)}  method.  The method
takes a query specification as argument, which is constructed with the
help of  another DSL, similar  to the one  used for sorting.  You have
access to a certain number of primitive queries, which are then linked
together  with  the  usual  boolean  operators, as  in  the  following
example:

\verbatiminput{code/search.py}

We first get the taxonomy  item corresponding to articles, and we then
compose  the   following  query:  get  all  the   documents  that  are
\textit{not} articles, and which contain the word \textit{laziness} in
any attribute.


\section{Importing and exporting}
\label{sec:importexport}

As  \pyblio  is not  bound  to a  single  data  schema, importing  and
exporting from  specific formats (like MARC,  BibTeX, Dublin Core,...)
cannot  be achieved  once  for all.  In  order to  avoid  the need  to
recreate a  BibTeX parser for every database  schema invented, \pyblio
makes a  clear separation between  \textit{syntactic parsers}, located
in  \py{Pyblio.Parsers.Syntactic}  and  \textit{semantic parsers},  in
\py{Pyblio.Parsers.Semantic}. A syntactic parser  is only in charge of
analyzing  or  generating  a   file  format,  without  any  assumption
regarding the meaning of the  fields it reads. These syntactic parsers
are then reused by the semantic code, which relates the meaning of the
fields to the corresponding database.

In addition, the parsers are  written so that the handling of separate
fields can be easily overridden in a subclass.  This makes it possible
to extend them or  take some local \textit{specificities} into account
(if you  need to  massage data that  contains systematic  errors, this
proves \textit{very} useful).

The  following example  assumes you  have created  a BibTeX-compatible
database, as explained in the section\vref{sec:registry}. It will then
open a proper BibTeX file, and merge it into the current database. The
list of imported references is returned as a result set.
\verbatiminput{code/import.py}


\section{Citation formatting}
\label{sec:citation}

The \textit{painful}  part of writing  citation formatting code  is to
take  into account incomplete  records (sometimes  you don't  know the
volume or  the pages,  or you  only know one  of the  two,...) without
multiplying explicit checks that would quickly be boring. In addition,
it is important to make it  easy to factor out common operations, like
formatting a list of authors, so  that you can reuse them in different
contexts.

\pyblio  provides a \textit{domain  specific language}  that addresses
these problems.   A domain specific language  (or DSL for  short) is a
language specifically intended to solve  a given problem, but which is
usually built up from a more  general language.  In our case, it means
that \pyblio provides a set  of classes, functions and constructs that
are highly  specialized to make  the business of writing  the citation
code  easy.  The beauty  of the  idea is  that, in  case of  a missing
feature in  this DSL, you still have  all the power of  python at your
fingertips.

In  any  case, this  DSL  is not  intended  as  a complete  formatting
language, so  you cannot use  it to lay  out your citations in  a full
blown HTML web  page for instance.  However, once  a citation is built
up from a record, the specific  part of putting it in a larger context
is comparatively easy.

Back to practice. You can define some citation fragments like this:

\begin{verbatim}
from Pyblio.Format import People, all, one

authors = People.lastFirst(all('author'))
title = one('title') | u'(no title)'
\end{verbatim}

In this example,  the \py{authors} variable is build  up by taking all
the values  in the author  field (\py{all('author')}), and  by passing
them through  the \py{lastFirst} function,  which will format  them as
\textit{Last Name, First Name}.  The \py{Person} module contains other
formatting variants for  person names if you want  to use initials for
instance.

The  \py{title} variable is  built by  taking the  first value  of the
title  field (via  the \py{one}  operator), and  in case  it  does not
exist,  by using the  string \textit{no  title} instead.   This \py{|}
alternative  operator can  be used  everywhere to  express  a fallback
value where a definition can be invalid.

You can then group the  authors and the title together, possibly while
adding some typographic style information in the process:
\begin{verbatim}
from Pyblio.Format import B

citation = join(', ')[B[title], authors]
\end{verbatim}

The \py{join} operator  will take the parts between  square braces and
bind them  together with the text  specified in parameter,  a comma in
that case.   When one of the  composing parts is not  available, it is
simply ignored, unless  no part is available, in  which case the whole
expression  is invalid  (which  can  be trapped  by  using the  \py{|}
operator). In the example, the title is enclosed in a bold \py{B} tag.

Once the citation style is defined, it must be \textit{compiled} on a
specific database:
\begin{verbatim}
formatter = citation(db)
\end{verbatim}

This operation checks  that all the fields accessed  are actually part
of the schema.  It also pre-computes certain information,  so that the
actual formatting of specific records can be faster.

Then, you can use the returned formatter and apply it to any number of
records from the corresponding database:
\begin{verbatim}
cited = formatter(record)
\end{verbatim}

You still  don't get a  definitive result, as  you need to  select the
output format  for your citation. If you  want it in HTML,  you can do
this last operation:
\begin{verbatim}
from Pyblio.Format import HTML

html = HTML.generate(cited)
\end{verbatim}

Now, \py{html} contains a properly escaped HTML fragment which you can
use in your own context.

\section{Querying external databases}

\pyblio has a standard interface for querying external databases, like
\textsl{PubMed} or the \textsl{Web  of Science}. These queries rely on
the asynchronous \py{twisted} library,  which makes it possible to run
such  a query  from a  graphical  interface without  blocking, and  to
interrupt it easily.

An example of such a query is described below.

\verbatiminput{code/query.py}

In this code, a query  object \py{wok} is created, which will directly
modify  the database  passed  in  parameter. Then  a  actual query  is
registered  by calling  \py{wok.search}.  This returns  two values:  a
\textsl{deferred object}, which is a twisted abstraction. You can then
\textit{plug} a callback, in  our case the \py{success()} function, to
be called  when the  search succeeds. The  second value returned  is a
result set,  which will  be filled with  the entries retrieved  by the
query.

Note that so far  the query did not run. To start  it, you need to run
\textsl{twisted}'s main loop, called  the reactor. This function won't
exit unless you call \py{reactor.stop()} somewhere. Please have a look
at            \textsl{twisted}'s            documentation\footnote{see
  \url{http://twistedmatrix.com/}} to  learn how  to make use  of this
powerful framework.


\section{Adapting to another schema}

In the preceding  example, we fetched results from  the Web of Science
in a database of type \py{org.pybliographer/wok/0.1}. This is a schema
specially crafted  for records coming from this  database, but usually
this  is  not what  you  expect  to store  in  your  own, say,  BibTeX
database.  So  what,  do  you  need to  create  another  query  engine
specially  for  your  database?   Fortunately  no,  you  can  use  the
\textsl{adaptation  mechanism} to  make this  Web of  Science database
look like a BibTeX database with a simple call like:

\begin{verbatim}
from Pyblio import Adapter

bibtex = Adapter.adapt_schema(
      db, 'org.pybliographer/bibtex/0.1')
\end{verbatim}

When this call succeeds, you will  get in return a new database called
\py{bibtex}, which  will contain everything contained  in \py{db}, but
in the BibTeX  schema. Please note that your  database is \textit{not}
duplicated, the \py{bibtex} database is just some kind of overlay that
behaves  as   a  normal  database,   but  uses  the   initial  content
dynamically.

Unfortunately, these adapters objects that  do the mapping do not come
out of  thin air,  and need to  be registered  in the system,  via the
usual \py{.rip} registry mechanism.

\chapter{Extending \pyblio}

\section{Specializing a parser}
\label{sec:specializing}

Let's say your BibTeX file uses a field called \py{status} that is not
standard.   You need  to create  a new  schema that  declares  it, and
derive the base BibTeX parser to provide an extra field handler:
\begin{verbatim}
class MyBibTeXReader(Semantic.BibTeX.Reader):
    def do_status_field(self, key, value):
        # do things for the field "status"
\end{verbatim}

Once this is done, you can register the new reader in a RIP file.

In  some cases,  it is  necessary  to perform  cross-field checks  and
modifications.  This can  be achieved  by using  the  following simple
extension hooks:
\begin{description}
\item[\py{Parser.record\_begin(self)}] is invoked  at the beginning of
  each record.
\item[\py{Parser.record\_end(self)}] is invoked once all the fields of
  a record have been parsed.
\item[\py{Parser.do\_default(self, field, value)}] will be invoked for
  unknown fields.
\end{description}

\section{Writing an external query engine}


\end{document}
