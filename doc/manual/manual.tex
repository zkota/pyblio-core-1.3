\documentclass[a4paper]{report} % -*- coding: latin-1 -*-

\usepackage{xspace, palatino, hyperref, graphicx, verbatim, varioref}
\usepackage[latin1]{inputenc}

\newcommand{\pyblio}{\textsl{pybliographer}\xspace}
\newcommand{\py}[1]{\texttt{#1}}
\newcommand{\file}[1]{\texttt{#1}}

\hyphenation{bib-li-o-gra-phic}

\title{Writing bibliographic tools with \pyblio}
\author{Fr�d�ric Gobry}

\begin{document}
\maketitle

\tableofcontents

\chapter{Introduction}

\pyblio   is   a   developer-oriented   framework   for   manipulating
bibliographic  data.  It  is written  in  \textsl{python}\footnote{see
  \url{http://python.org/}}, and  uses extensively the  dynamic nature
of this language.

\pyblio  does   not  try  to   define  another  standard   format  for
bibliographic  data, nor  does it  solely  rely on  a single  existing
standards.    Standards  are   important   in  order   to  allow   for
interoperability and durability.  Unfortunately, real-world data often
contain  a  great  number   of  mistakes,  or  reflect  certain  local
conventions. \pyblio is on  the \textit{pragmatic} side of considering
these issues as part of its business: most of the parsing tasks can be
easily overriden and  specialized in order to \textit{fit  the code to
  the data}, and not the other way around.


\section{Basic concepts}

\pyblio  deals  with  sets  of  \py{Record}s, stored  in  a  so-called
\py{Database}.  This database can  be actually  implemented on  top of
different systems. Two are available  today, one based on a single XML
file,  using  a  custom  XML  dialect, the  other  based  on  Berkeley
DB\footnote{see  \url{http://www.sleepycat.com/}},  a  very  efficient
database system.

Each record represents an elementary  object you want to describe, and
has  a  number  of  \textsl{attributes}.   For instance,  if  you  are
describing a  book, one attribute will be  its \textsl{title}, another
its \textsl{ISBN},  etc. Each of  these attributes can contain  one or
more  values,  all  of   the  same  \textsl{type}.   To  continue  the
description  of  our  book,   we  probably  have  the  \textsl{author}
attribute,  which contains  as many  \py{Person} values  as  there are
authors for the  book. All the values of a given  attribute are of the
same type.

In some cases, simply having  this flat key/value model to describe an
object  is not  enough.   \pyblio  allows, for  every  value of  every
attribute, to  provide a set of  \textsl{qualifiers}. These qualifiers
are also attributes which can hold one or more values.  If my book, or
information  about the  book, is  available  via the  internet, I  can
provide  a  \textsl{link}  attribute,  but  for  each  of  the  actual
\py{URL}s  provided,  I  might  wish  to  add  a  \textsl{description}
qualifier, which will indicate, say, if the URL points to the editor's
website, or to a review, etc.

This nesting of objects is best described in
figure~\ref{fig:hierarchy}.

\begin{figure}[htbp]
  \centering
  \includegraphics[width=0.9\textwidth]{fig/hierarchy.pdf}
  \caption{Objects manipulated in \pyblio}
  \label{fig:hierarchy}
\end{figure}

\pyblio comes with a set of defined attribute types, like \py{Person},
\py{Text}, \py{Date}, \py{ID} (see the \py{Pyblio.Attribute} module
for a complete list), and can be extended to support your own types.


\subsection{The database schema}
\label{sec:schema}

Even though  attributes are typed,  the data model described  above is
quite flexible.  In  order for \pyblio to help  you checking that your
records are properly  typed, it needs to know  the database schema you
are  using.  This  schema,  usually stored  in  an XML  file with  the
extension \file{.sip},  simply lists  the known attributes  with their
type and  the qualifiers it  allows for its values.   Some \file{.sip}
files  are  distributed   with  \pyblio,  and  can  be   seen  in  the
\py{Pyblio.RIP} directory.

In   addition   to  validation   information,   the  schema   contains
human-readable  description  of  the  different  fields,  possibly  in
several languages, so  that it can be automatically  extracted by user
interfaces to provide up-to-date information.

\subsection{Taxonomies}
\label{sec:taxonomies}

Taxonomies can be used as \textit{enumerated values}, say for listing
the possible types of a document, or the language in which a text is
written. They have however the extra capability of being hierarchical:
you can define subcategories of a main category.  For instance,
imagine a \py{doctype} taxonomy with the following values:

\begin{itemize}
\item Published
  \begin{itemize}
  \item Article
    \begin{itemize}
    \item Peer-reviewed
    \item Non peer-reviewed
    \end{itemize}
  \item Book
  \end{itemize}
\end{itemize}

You can tag an article as \py{Peer-reviewed}, but in the case you
don't know its status, you can use \py{Article}. If you search for all
the \py{Published} documents, you will retrieve all those that have
the \py{Published} tag, but also those that are articles (either
peer-reviewed or not), books,...

\pyblio uses the \py{Pyblio.Attribute.Txo} object to
\textit{represent} a logical value in a given taxonomy. Then, a record
can be tagged with this \py{Txo} object by adding a
\py{Pyblio.Attribute.TxoItem} value in the corresponding attribute.

Taxonomies can be declared and pre-filled in a database schema, so
that any database created from the schema will at least contain the
specified taxonomies.

To see how these taxonomies can be further created and modified,
please have a look at the \py{txo} member of a \py{Database} object,
which is an instance of the \py{Pyblio.Store.TxoGroup} class.

\subsection{Result sets}
\label{sec:resultsets}

Result sets are used to manipulate an explicit list of records, among
all the records kept in a database. They are returned from queries on
the database, and can be manipulated by the user. Result sets are
somewhat like mathematical sets, as you cannot put duplicate values in
them, and they have no default ordering of their elements. You can
create result sets via the \py{rs} attribute of your database, which
is an instance of the \py{Pyblio.Store.ResultSetStore}.

A special result set is available as \py{Pyblio.entries}, and contains
at every time \textbf{all} the records of the database.


\subsection{Views}
\label{sec:views}

We have seen that result sets are \textbf{not} ordered. However, in
many cases, one needs to provide the records in a specific order. To
do so, you can create a \textsl{view} on top of a result set. This
view is created by calling the \py{view} method of the result set,
with an \py{order} parameter being the description of the sort order
you wish to have. The module \py{Pyblio.Sort} provides elementary
constructs to build such a description.

Once the view is created, modifying the corresponding result set leads
to updating the view accordingly.

\section{Manipulating data}
\label{sec:manipulating}

This section describes some simple operations you can perform on some
subset of a \pyblio database.

\subsection{Loading and saving}
\label{sec:loading}

The first thing you need to do is of course \textit{actually having} a
database available. The following code does the job:
\verbatiminput{code/create.py}

This example relies on the fact that you already have a schema at
hand. There are schemas available in the \file{Pyblio.RIP} directory.
It the starts by reading the schema. The next step is to select the
actual physical store which will hold your database. We choose to
store it in a simple XML file, whose canonical extension is
\file{.bip}. The last operation actually creates the database with the
specified schema.

Independently  of  the  selected  store,  it  is  always  possible  to
\textit{export} a  database in the \file{.sip} format,  by calling the
\py{db.xmlwrite(...)} method of the database.  Such a file can then be
reused   later  on  by   using  \py{store.dbimport(...)}   instead  of
\py{store..dbcreate(...)}.

When you have finished modifying your database, you can call
\py{db.save()} method to ensure that it is properly saved.

\textbf{Caution:} the bsddb store for instance is updated at every
actual modification, not only when you call the \py{save} method.
Don't rely on it to provide some kind of \textsl{rollback} feature.

\subsection{Using the registry}
\label{sec:registry}

\pyblio has a mechanism to register known schemas, and specify which
import and export filters can properly work with each schema. This
mechanism can be used to create our database by asking for a specific
schema, as shown below:
\verbatiminput{code/registry.py}

The registry must be first initialized. Then you can ask for a
specific schema, in that case a schema that supports BibTeX
databases.

\subsection{Updating records}
\label{sec:updating}

The next example will loop over all the records in a database, and add
a new author to the list of authors.
\verbatiminput{code/addauthor.py}

We use the \py{itervalues()} iterator to loop over all the records
stored in the database. Then, we simply insert a new value in the
\py{author} attribute. The \py{record.add(...)} method takes care of
creating the attribute if it does not exist yet.

One thing not to forget is to store the record back in the database
once the modification is performed. Without this step, you might
experience weird behavior where some modifications are not properly
kept.

We finish by saving the database.

\subsection{Sorting}
\label{sec:sorting}

To sort records, you create \textit{views} (see
section~\vref{sec:views}). You can of course create multiple views on
top of a single result set. In order to sort the whole database,
simply create the view on \py{database.entries} instead of a result
set. If you want to sort your database by decreasing year and then by
author, you can use a view like that: 

\verbatiminput{code/sort.py}

So, sorting  constraints can be  arbitrarily chained with  the \py{\&}
operator,and  each constraint  can be  either  \textit{ascending} (the
default),  or   \textit{descending}.   This  defines   a  very  simple
\textsl{Domain Specific  Language}, or  DSL for short.  Such languages
also appear in other part of \pyblio (searching, citation formatting),
as they are  a convenient way to describe  complex abstraction without
having to reinvent a complete environment.

\subsection{Searching}
\label{sec:searching}

To search,  you call  the \py{database.query(...)} method.  The method
takes a query specification as argument, which is constructed with the
help of another DSL, similar to the one used for sorting. You have
access to a certain number of primitive queries, which are then linked
together with the usual boolean operators, as in the following
example:

\verbatiminput{code/search.py}

We first get the taxonomy  item corresponding to articles, and we then
compose  the   following  query:  get  all  the   documents  that  are
\textit{not} articles, and which contain the word \textit{lazyness} in
any attribute.


\section{Importing and exporting}
\label{sec:importexport}

TODO


\section{Citation formatting}
\label{sec:citation}

TODO



\chapter{Extending \pyblio}

TODO


\section{Specializing a parser}
\label{sec:specializing}

TODO


\end{document}
